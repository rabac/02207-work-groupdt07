\documentclass[11pt,a4paper]{article}

\usepackage{url,,}
\usepackage{graphicx}
\usepackage{hyperref}
\usepackage{amsfonts}
\usepackage{amssymb}
\usepackage{amsmath}
\usepackage{multirow}
\usepackage{listings}
\usepackage{fullpage}
\usepackage{fancyhdr,a4wide}
\usepackage{makeidx}
\usepackage{placeins}
%\usepackage[procnames,noindent]{lgrind}

\lstset{ %
language=VHDL,                % choose the language of the code
basicstyle=\footnotesize,       % the size of the fonts that are used for the code
showstringspaces=false,         % underline spaces within strings
%numbers=left,                   % where to put the line-numbers
%numberstyle=\footnotesize,      % the size of the fonts that are used for the line-numbers
%stepnumber=1,                   % the step between two line-numbers. If it's 1 each line will be numbered
%numbersep=5pt,                  % how far the line-numbers are from the code
%backgroundcolor=\color{white},  % choose the background color. You must add \usepackage{color}
showspaces=false,               % show spaces within strings adding particular underscores
showtabs=false,                 % show tabs within strings adding particular underscores
escapeinside={\%*}{*)}          % if you want to add a comment within your code
}

\begin{document}	

\begin{titlepage}

\thispagestyle{fancy}
\lhead{}
\chead{
\large{\textit{
Informatics and Mathematical Modelling\\
Technical University of Denmark}}}
\rhead{}
\rule{0pt}{50pt}
\vspace{3cm}

\begin{center}
 	\huge{\textbf{02207 : Advanced Digital Design Techniques}}\\
 	\vspace{1cm}
 	\huge{Low-pass Filter (2 x 1-D)}\\
 	\vspace{1cm}
 	\huge{\textit{Examination Project}}\\
 	\vspace{1cm}
 	\huge{Group \textit{dt07}}\\
\end{center}

\vspace{4cm}

\begin{flushright}
	\LARGE{Markku Eerola (s053739)}\\
	\vspace{0.3cm}
	\LARGE{Rajesh Bachani (s061332)}\\
	\vspace{0.3cm}
	\LARGE{Josep Renard (s071158)}\\
\end{flushright}
\cfoot{\today}
\end{titlepage}

\newpage 
\tableofcontents

\newpage

\section{Introduction}
The project that we have implemented is a 2x1D filter of size 3x3 for convolution of an image of size 256x256 pixels.  

\section{Design Architecture}
%give the block design here. we need the following figure:
%a complete diagram of the processor - indicating the input and output signals, and also containing the internal blocks like the cache register,filter register, fsm_in, the multipliers, adders, mux, and fsm_out.

%here we should mention the details about the data path, bit widths etc. 
%we should give a brief explanation of each of the components in the architecture - just a couple of lines explanation should be good enough. 
%also we should explain the purpose of each of the input and output signals of the memory and the processor.

\section{Sequencing of Operations}
%explain the entire sequence in which the image is convoluted. in particular
\subsection{Memory Initialization}
%how the initialization is done - the input FSM keeps the write of the input memory till 65526 clock cycles - so all the pixels are read from the hex file.
%in the meantime, the output FSM keeps the write signal of the output memory to 1 and it is initialized with 0.
\subsection{Memory Read and Write by Processor}
%Once the initialization is done, 
% 1. the input fsm reads 3 pixels - and then waits for 9 clock cycles
% 2. the output fsm starts when the input fsm becomes idle - and runs the sequence read-idle-write 3 times.
%give picture from modelsim here showing how the two fsms synchronize.
\subsection{Memory Access Sequence}
%mention the sequence of the memory access from the input fsm and the output fsm

\section{Finite State Machines}
\subsection{Input Controller}

%state machine diagram
\subsection{Output Controller}

%state machine diagram

\section{Synthesis}
%present the results from the various reports here. 
% the critical path
% area 
% power dissipation

\section{Results}
%put the convoluted and the original image here to show the results.

\section{Discussion}
%explain how would the design change if the filter size is changed. parameters that would change:
%number of adders and multipliers.

\end{document}