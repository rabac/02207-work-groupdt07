\documentclass[11pt,a4paper]{article}

\usepackage{url}
\usepackage{graphicx}
\usepackage{amsfonts}
\usepackage{amssymb}
\usepackage{amsmath}
\usepackage{amsfonts}
\usepackage{amssymb}
\usepackage{amsmath}
\usepackage{multirow}
\usepackage{listings}
\usepackage{fullpage}
\usepackage{fancyhdr,a4wide}
\usepackage{makeidx}
\usepackage{placeins}
\usepackage[procnames,noindent]{lgrind}

\lstset{ %
language=VHDL,                % choose the language of the code
basicstyle=\footnotesize,       % the size of the fonts that are used for the code
showstringspaces=false,         % underline spaces within strings
%numbers=left,                   % where to put the line-numbers
%numberstyle=\footnotesize,      % the size of the fonts that are used for the line-numbers
%stepnumber=1,                   % the step between two line-numbers. If it's 1 each line will be numbered
%numbersep=5pt,                  % how far the line-numbers are from the code
%backgroundcolor=\color{white},  % choose the background color. You must add \usepackage{color}
showspaces=false,               % show spaces within strings adding particular underscores
showtabs=false,                 % show tabs within strings adding particular underscores
escapeinside={\%*}{*)}          % if you want to add a comment within your code
}

\begin{document}	

\begin{titlepage}

\thispagestyle{fancy}
\lhead{}
\chead{
\large{\textit{
Informatics and Mathematical Modelling\\
Technical University of Denmark}}}
\rhead{}
\rule{0pt}{50pt}
\vspace{3cm}

\begin{center}

 	\huge{\textbf{02207 : Advanced Digital Design Techniques}}\\
 	\vspace{1cm}
 	\huge{Lab 1: Exercise on Synthesis}\\
 	\vspace{1cm}
 	\huge{Group \textit{dt07}}\\	
\end{center}

\vspace{4cm}

\begin{flushright}
	\LARGE{Markku Eerola (s053739)}\\
	\vspace{0.3cm}
	\LARGE{Rajesh Bachani (s061332)}\\
	\vspace{0.3cm}
	\LARGE{Josep Renard (s071158)}\\
\end{flushright}
\cfoot{\today}
\end{titlepage}

%\begin{abstract}
%\centering
%Abstract to be created.
%\end{abstract}

%-----------------------------------------------------------
\newpage 
\tableofcontents

\newpage 
\section{Purpose of the Exercise}

The goal of the exercise was to get familiar with the process of synthesis of digital circuits, using special tools for synthesis such as Design Vision. A register-level netlist containing a 24-bit adder is synthesized in the exercise, for different values of the clock time period. The reports concerning the timing constraints, area, power consumption etc. are documented in the report.

\section{Behavioral Description for 24-bit Adder}

The VHDL code for a simple 24-bit adder is provided below.\\ 

\lstinputlisting[frame=trbl]{../source/NBitAdder.vhdl}

% include the VHDL file %

The behavior of this adder is verified in Modelsim. 

\section{Behavioral Description for 24-bit Register}


\section{Top-Level Netlist from 24-bit Adder and 24-bit Register}

 
\section{Synthesis Results}


\end{document}