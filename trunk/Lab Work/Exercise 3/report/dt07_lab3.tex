\documentclass[11pt,a4paper]{article}

\usepackage{url,,}
\usepackage{graphicx}
\usepackage{hyperref}
\usepackage{amsfonts}
\usepackage{amssymb}
\usepackage{amsmath}
\usepackage{amsfonts}
\usepackage{amssymb}
\usepackage{amsmath}
\usepackage{multirow}
\usepackage{listings}
\usepackage{fullpage}
\usepackage{fancyhdr,a4wide}
\usepackage{makeidx}
\usepackage{placeins}
\usepackage[procnames,noindent]{lgrind}

\lstset{ %
language=VHDL,                % choose the language of the code
basicstyle=\footnotesize,       % the size of the fonts that are used for the code
showstringspaces=false,         % underline spaces within strings
%numbers=left,                   % where to put the line-numbers
%numberstyle=\footnotesize,      % the size of the fonts that are used for the line-numbers
%stepnumber=1,                   % the step between two line-numbers. If it's 1 each line will be numbered
%numbersep=5pt,                  % how far the line-numbers are from the code
%backgroundcolor=\color{white},  % choose the background color. You must add \usepackage{color}
showspaces=false,               % show spaces within strings adding particular underscores
showtabs=false,                 % show tabs within strings adding particular underscores
escapeinside={\%*}{*)}          % if you want to add a comment within your code
}

\begin{document}	

\begin{titlepage}

\thispagestyle{fancy}
\lhead{}
\chead{
\large{\textit{
Informatics and Mathematical Modelling\\
Technical University of Denmark}}}
\rhead{}
\rule{0pt}{50pt}
\vspace{3cm}

\begin{center}

 	\huge{\textbf{02207 : Advanced Digital Design Techniques}}\\
 	\vspace{1cm}
 	\huge{Exercise of Retiming}\\
 	\vspace{1cm}
 	\huge{\textit{LAB 3}}\\
 	\vspace{1cm}
 	\huge{Group \textit{dt07}}\\
\end{center}

\vspace{4cm}

\begin{flushright}
	\LARGE{Markku Eerola (s053739)}\\
	\vspace{0.3cm}
	\LARGE{Rajesh Bachani (s061332)}\\
	\vspace{0.3cm}
	\LARGE{Josep Renard (s071158)}\\
\end{flushright}
\cfoot{\today}
\end{titlepage}

%\begin{abstract}
%\centering
%Abstract to be created.
%\end{abstract}

%-----------------------------------------------------------
\newpage 
\tableofcontents

\newpage 
\section{Introduction}
\subsection{Retiming}
% to just explain what retiming is - the concept and its purpose. that we need to retime circuits manually - so that the synthesizer could identify two different paths easily and perform low power synthesis for the non critical path.
\subsection{Simple Design for Division}
%explain briefly the circuit for division. mention about the critical path in the circuit.
\subsection{Design for Division using Retiming}
%explain in detail the circuit for division using retiming. we should argue what is expected from this, and how we should be able to save power in this.
\subsection{Authors by Section}
\begin{itemize}
\item \textit{Rajesh Bachani} 
\item \textit{Josep Renard} 
\item \textit{Markku Eerola} 
\end{itemize}

\section{Implementation of Division using Retiming}
\label{section:impl}
%explain specifically the changes that were implemented for using the retimed circuit for division. this could also include parts of the VHDL, but mostly it should concentrate on highlighting the changes in terms of the connections between the components.
%we would not have any section for implementation, giving the code, since there is a lot of code in the whole thing. so we should try and just explain all the changes in the code here only.
\section{Power Report and Cell Count}
\label{section:power}
% put the reports from the power analysis of both the designs here. also, give a short recap kind of a table here. also, write about the cells here.
\section{Discussion}
\label{section:discussion}
%discuss the report at length here. we should clearly justify the results, by explaining why the SVT cells count has reduced and HVT has increased, and why the cell internal power has reduced. ofcourse this was expected, but nice explanation is needed here. 
\end{document}