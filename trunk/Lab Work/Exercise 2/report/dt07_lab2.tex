\documentclass[11pt,a4paper]{article}

\usepackage{url,,}
\usepackage{graphicx}
\usepackage{amsfonts}
\usepackage{amssymb}
\usepackage{amsmath}
\usepackage{amsfonts}
\usepackage{amssymb}
\usepackage{amsmath}
\usepackage{multirow}
\usepackage{listings}
\usepackage{fullpage}
\usepackage{fancyhdr,a4wide}
\usepackage{makeidx}
\usepackage{placeins}
\usepackage[procnames,noindent]{lgrind}

\lstset{ %
language=VHDL,                % choose the language of the code
basicstyle=\footnotesize,       % the size of the fonts that are used for the code
showstringspaces=false,         % underline spaces within strings
%numbers=left,                   % where to put the line-numbers
%numberstyle=\footnotesize,      % the size of the fonts that are used for the line-numbers
%stepnumber=1,                   % the step between two line-numbers. If it's 1 each line will be numbered
%numbersep=5pt,                  % how far the line-numbers are from the code
%backgroundcolor=\color{white},  % choose the background color. You must add \usepackage{color}
showspaces=false,               % show spaces within strings adding particular underscores
showtabs=false,                 % show tabs within strings adding particular underscores
escapeinside={\%*}{*)}          % if you want to add a comment within your code
}

\begin{document}	

\begin{titlepage}

\thispagestyle{fancy}
\lhead{}
\chead{
\large{\textit{
Informatics and Mathematical Modelling\\
Technical University of Denmark}}}
\rhead{}
\rule{0pt}{50pt}
\vspace{3cm}

\begin{center}

 	\huge{\textbf{02207 : Advanced Digital Design Techniques}}\\
 	\vspace{1cm}
 	\huge{Design for Low Power by Reducing Switching Activity}\\
 	\vspace{1cm}
 	\huge{\textit{LAB 2}}\\
 	\vspace{1cm}
 	\huge{Group \textit{dt07}}\\
\end{center}

\vspace{4cm}

\begin{flushright}
	\LARGE{Markku Eerola (s053739)}\\
	\vspace{0.3cm}
	\LARGE{Rajesh Bachani (s061332)}\\
	\vspace{0.3cm}
	\LARGE{Josep Renard (s071158)}\\
\end{flushright}
\cfoot{\today}
\end{titlepage}

%\begin{abstract}
%\centering
%Abstract to be created.
%\end{abstract}

%-----------------------------------------------------------
\newpage 
\tableofcontents

\newpage 
\section{Introduction}

The purpose of this exercise was to estimate the power dissipation in a digital circuit due to the switching activity in the cells. Dynamic power dissipated in a digital circuit is the sum of two termsthe power spent in 

\subsection{Authors by Section}
\begin{itemize}
\item \textit{Markku Eerola} 
\item \textit{Josep Renard} 
\item \textit{Rajesh Bachani} 
\end{itemize}

\section{Designs for Serial to Parallel Conversion}
In this section, we give an overview of the three designs for serial to parallel conversion, which are evaluated for their power consumption in this exercise.

\subsection{Design A: Shift Register}

\subsection{Design B: Register with Enable}

\subsection{Design C: Register with Clock-Gating}


\section{Simulation of the designs with Modelsim}
All the three designs are simulated with Modelsim, to verify the functionality. 




\section{Power Reports from Design Vision}


\section{Discussion on the Reports}


\section{Implementation}


\end{document}